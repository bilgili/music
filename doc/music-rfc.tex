\documentclass[a4paper]{report}

\usepackage[numbers]{natbib}
\citestyle{newapa}
\usepackage[utf8]{inputenc}
\usepackage{fancyhdr}
\usepackage{color}
\usepackage{graphicx}
\usepackage{listings}
\usepackage{makeidx}
\usepackage{comment}

%%%%% Formatting %%%%%

% Use the metatext environment around text that should not appear in
% the final document
%\newenvironment{metatext}%
%{\color{blue}}%
%{}
\excludecomment{metatext}

% Use the rationale environment around arguments for design decisions
\newenvironment{rationale}%
{\par\begin{quote}\textbf{Rationale:}}%
{\par\end{quote}}


% Use the head environment around method heads
\lstnewenvironment{head}[1]%
{\lstset{frame=topline,emph={#1},emphstyle=\color{blue}\textbf}}%
{}


% Use the parameters environment after heads
\newenvironment{parameters}%
{\begin{tabular}{@{\hspace{2em}}lp{0.6\textwidth}}}%
{\end{tabular}\par\vspace{1mm}\par\hrule\par\vspace{5mm}}


% Use the code environment around method code examples
\lstnewenvironment{code}[1]%
{\lstset{frame=single,caption={#1}}}%
{}

\renewcommand{\lstlistingname}{Example}

% Use the responsible command to indicate which author is responsible
% for the present section
\newcommand{\responsible}[1]%
{{\color{red}[#1 is responsible for this section]}}


\fancyhead{}
\fancyhead[L]{\slshape\leftmark}

\pagestyle{fancy}
\makeindex

%%%%% Actual content starts here %%%%%
\begin{document}

\lstset{language=C++}

\title{MUSIC --- Multi-Simulation Coordinator\\[2ex]
  Request For Comments\\}

\author{Örjan Ekeberg and Mikael Djurfeldt}

\maketitle

\begin{abstract}
  MUSIC is an API allowing large scale neuron simulators using MPI
  internally to exchange data during runtime.  MUSIC provides
  mechanisms to transfer massive amounts of event information and
  continuous values from one parallel application to another.  Special
  care has been taken to ensure that existing simulators can be
  adapted to MUSIC.  In particular, MUSIC handles data transfer
  between applications that use different time steps and different
  data allocation strategies.
\end{abstract}


\tableofcontents

\listoffigures

\chapter{Introduction}

This document constitutes a preliminary specification for the
multi-simulation coordinator MUSIC.  The main purpose of the current
document is to make it possible for potential users of MUSIC to
comment on the design before the full implementation is finalized.

\section{Scope}

MUSIC is a standard for run-time exchange of data between parallel
applications in a cluster environment.  The standard is designed
specifically for interconnecting large scale neuronal network
simulators, either with each-other or with other tools.

A typical usage example is illustrated in figure~\ref{fig:multisim},
where three applications ($A$, $B$, and $C$) are executing in parallel
while exchanging data via MUSIC.  We will refer to this as a
\emph{multi-simulation}, since the participating applications
typically are neuronal simulators, or tools to support such
simulators.  In this example, application $A$ produces runtime data
which is then used by $B$ and $C$.  In addition, $B$ and $C$ mutually
send data to each other.  The data sent between applications can be
either event based, such as neuronal spikes, or graded continuous
values, for example membrane voltages.

\begin{figure}
  \begin{center}
    \includegraphics[width=0.5\textwidth]{figures/multisim}
    \caption[Typical multi-simulation]{\label{fig:multisim}
      Illustration of a typical multi-simulation using MUSIC.  Three
      applications, $A$, $B$, and $C$, are exchanging data during
      runtime.
    }
  \end{center}
\end{figure}

The primary objective of MUSIC is to support multi-simulations where
each participating application itself is a parallel simulator with the
capacity to produce and/or consume massive amounts of data.  This
promotes \emph{inter-operability} by allowing models written for
different simulators to be simulated together in a larger system.  It
also enables \emph{re-usability} of models or tools by providing a
standard interface.  The fact that data is spread out over a number of
processors makes it non-trivial to coordinate the transfer of data so
that it reaches the right destination at the right time.  The task for
MUSIC is to relieve the applications from handling this complexity.


\section{Design Goals}

\subsection{Portability}

The MUSIC library and support software have been designed to run
smoothly on state-of-the-art high-performance hardware.  For maximal
portability, the software is written in C++, which is the
de facto standard for current high-end hardware.  MUSIC also provides a
pure C-interface, making it possible for applications written i C or
FORTRAN to participate in a MUSIC multi-simulation.

Most, if not all, current efforts in large scale neuronal simulations
are based on the MPI\index{MPI} standard.  MUSIC is built on top of
MPI, and uses it to run the different simulators.  MUSIC provides
means to allow each simulator to use MPI internally without
interfering with the others.

During the development of MUSIC we have used two reference platforms:
Intel-based multi-core workstations and the IBM
BlueGene/L\index{BlueGene/L}\index{IBM BlueGene} supercomputer.  These
platforms can be considered as two extremes, where the multi-core
machine represents a small parallel environment while the BlueGene/L
represents a large scale parallel supercomputer with special
requirements.  In particular, the compute nodes on the BlueGene/L do
not support multiple threads or processes.


\subsection{Simplicity}

For MUSIC to be useful, it must be possible to adapt existing
simulators so that they can participate in a multi-simulation without
too much effort.  We rely on the simulator developers to make these
adaptations.  An important design goal has therefore been to adapt the
design to the typical structure of current simulators.  It should be
possible to add usage of the MUSIC library without invasive
restructuring of the existing simulator.

The requirements on an application using MUSIC is primarily that it,
during the setup phase, declares what data should be exported and
imported, and that it repeatedly calls a function at regular intervals
during the simulation to allow MUSIC to make the actual data transfer.


\subsection{Independence}

The MUSIC interface ensures that each individual application does not
need special adaptation to specific properties of other applications.
The application only needs to adhere to the specification of the MUSIC
interface in order to communicate with other applications performing
complementary tasks.  This makes the development of MUSIC-aware
software independent of what other applications it will communicate
with.

We hope that this will facilitate the development of general purpose
tools.  For example, a researcher can develop a tool for calculating
synthetic EEG from simulation data.  Via MUSIC, this tool should then
be useful for anybody using any neuronal simulator which supports the
common MUSIC interface.


\subsection{Performance}

The MUSIC API has been designed to allow for data transport of high
bandwidth and low latency within the cluster.  One means of ensuring
the best use of the hardware while maintaining portability is to use
the facilities of MPI for communication.  MPI encapsulates software
optimizations for specific hardware. By basing the interface on MPI we
can benefit from such optimizations.


\subsection{Extensibility}

Where possible, MUSIC allows for extensions by the application
programmer.  Some classes in the MUSIC API (such as the index and data
maps) can be subclassed in order to provide facilities not available
directly in the API.


\section{Terminology}

\begin{description}
\item[application] We use the term
  \emph{application}\index{application} to denote a simulator or other
  program interfaced to MUSIC.  Each application is a parallel
  program, normally running on several processors.

\item[multi-simulation] We use the term
  \emph{multi-simulation}\index{multi-simulation} to refer to the
  whole parallel execution of multiple applications coordinated by
  MUSIC.

\item[port] Each application declares its ability to produce and
  consume data by publishing \emph{ports}\index{port}.  Ports are
  named by the application and provided with information about the
  datatype (continuous data, spike events, messages) and mapping onto
  different processors.  Ports are either
  \lstinline|input_ports|\index{input port} or
  \lstinline|output_ports|\index{output port}.

\item[connection] During the setup phase, MUSIC connects pairs of
  ports together to form \emph{connections}\index{connection}.  During
  the runtime phase, data is transferred over the connection from the
  producer of the data to the consumer.  While an
  \lstinline|input_port| can have only one connection, an
  \lstinline|output_port| can be connected to multiple
  \lstinline|input_ports|.

\item[data map] A data map\index{data map} denotes the information on
  where data actually resides within the application.  This is
  typically stored internally in the port data structure.  Data to be
  transferred over a connection can be regarded as a large array
  distributed over multiple processors. The data map tells on
  what processor each data element resides and how it should be
  accessed.

\item[ticks] During the runtime phase, all processes in each
  application must make a \emph{tick}\index{tick} call at regular
  intervals in simulated time.  At these tick points, MUSIC is allowed
  to use MPI to transfer data between processors.
\end{description}


\section{Relation to Existing Software}

MUSIC is not the only software project aiming to support
inter-operability between neural simulators.  In this section we will
briefly describe some related projects and specifically focus on how
they relate to MUSIC.

\paragraph{PyNN}\index{PyNN}

PyNN is a Python package for simulator-independent specification of
neuronal network models.  It provides a low-level procedural API and a
high-level object-oriented API.  Neuronal network models which are
specified using these API:s can be simulated on simulators supporting
PyNN, such as Neuron and NEST.

PyNN could be extended to support multi-simulations using the MUSIC
library.  Such an extension would provides means for controlling the
interaction between the simulator and the MUSIC library and would, for
example, support publishing of named ports.

It is possible, in principle, to write Python code to directly handle
communication between applications in a cluster, but such a solution
would be inefficient compared to using MUSIC, and might, in the end,
have to address the same problems which MUSIC provides a solution to.

\paragraph{Neurospaces}\index{Neurospaces}

The Neurospaces project promotes inter-operability and re-usability
through the development of independent software components, some of
which, together, will provide one of two alternative cores of the
Genesis 3 simulator.  One of the components, the Neurospaces Model
Container abstracts model description from the solver.  Another
component, the Discrete Event System can handle distribution and
queuing of spikes.  Components adhere to the CBI simulator
architecture.

It is possible to develop a MUSIC adapter consistent with the CBI
simulator architecture.  This would allow the Neurospaces framework,
and Genesis 3, to interface to independently running applications in a
cluster environemnt.

\begin{metatext}
\paragraph{Neosim}\index{Neosim}

\paragraph{MOOSE}\index{MOOSE}
\end{metatext}

\chapter{Execution Model}

\section{Phases of Execution}

A multi-simulation, i.e. a set of interconnected parallel
applications, is executed in three distinct phases:
\begin{description}
\item[\textbf{Launch}]\index{launch phase} is the phase where all the
  applications are started on the processors.  During this phase,
  MUSIC is responsible for distributing and launching the application
  binaries on the set of MPI processes allocated to the MUSIC job.
  Since MPI can be initialized first when the applications have been
  launched, most of this work needs to be performed outside of MPI.
  In particular, the tasks of accessing the command line argument of
  the MUSIC launch utility and of determining the ranks of processes
  before MPI initialization therefore has to be handled separately for
  different MPI implementations.

  Technically, the launch phase begins when \texttt{mpirun} launches
  the MUSIC binary and ends when the setup object constructor
  returns.  (See further description below.)

\item[\textbf{Setup}]\index{setup phase} is the phase when all
  applications can publish what ports they are prepared to handle
  along with the time step they will use and where data will be
  present (where in memory and/or on what processor).  During the
  setup phase, applications can read configuration parameters
  communicated via the common configuration file.  At the end of the
  setup phase, MUSIC will establish all connections.

  The setup phase begins when the setup object has been created and
  ends when the runtime object constructor returns.

\item[\textbf{Runtime}]\index{runtime phase} is the phase when
  simulation data is actually transferred between applications.  Via
  \texttt{tick} calls the simulated time of applications is
  kept in order.

  The runtime phase begins when the runtime object has been created
  and ends when the runtime object is destroyed.
\end{description}

From the application programmers point of view, these phases are
clearly separated through the use of two main components of the
MUSIC interface: the \emph{setup} and the \emph{runtime} object.  The
launch phase is not visible for the application since it handles the
situation before the application starts.

When the application initializes MUSIC at the beginning of execution
it receives a specific \emph{setup object}.  This object gives access
to the functionality relevant during the setup phase via its methods.
When done with the setup, the application program makes the transition
to the runtime phase by passing the setup object as an argument to the
\emph{runtime object} constructor which destroys the setup object.
The runtime object provides methods relevant during the runtime phase
of execution.

\section{Spatial Distribution of Data}
\label{sec:spatialdist}

Communication between applications is handled by ports.  Ports are
named sources (output ports) or sinks (input ports) of data flows.
The data to be communicated between the sender application and the
receiver may be differently organized in process memory, the
applications may run on different numbers of processes, and, the data
may be differently distributed among the sender processes and the
receiver processes, as is shown in Figure~\ref{fig:datamapping}.  How
does MUSIC know which data to send where?

In MUSIC, there are two views of the data to be communicated over a
connection.  Data elements are enumerated differently according to
these views.  MUSIC uses \emph{global indices}\index{global index} to
enumerate the entire set of data to be sent over the connection while
\emph{local indices}\index{local index} enumerate the subset of data
which is stored in the memory of a particular MPI process.  Data does
not need to be ordered in the same way according to the two views.
For example, local data stored in an array may be associated with an
arbitrary subset of global indices in an arbitrary order.

The MUSIC library is informed about the relationship between global
and local indices during the setup phase.  Two abstractions are used
to carry this information:

The \lstinline|index_map| maps indices local to global indices.  That
is, the \lstinline|index_map| tells which parts of a distributed data
array are handled by the local process and how the data elements are
locally ordered.

The \lstinline|data_map| encapsulates how a port accesses its data.
The \lstinline|data_map| contains an \lstinline|index_map|.  While an
index map is a mapping between two kinds of indices, the data map also
contains information about where in memory data resides, how it is
structured, and, the type of the data elements.

During setup every process of the application individually provides
the port with a \lstinline|data_map| (or an \lstinline|index_map| in
the case of event ports).

\begin{figure}
  \begin{center}
    \includegraphics[width=0.7\textwidth]{figures/datamapping}
    \caption[Mapping of data]{\label{fig:datamapping}
      Data transfer over a connection from an application running in
      four processes to an application running in three other.  The
      light gray areas in the sender and receiver represents the MUSIC
      port.  Dashed lines divide the application into distinct
      processes.
    }
  \end{center}
\end{figure}

\begin{rationale}
  While connections are often used to handle the transfer of spikes
  from one group of neurons to another, they should not be regarded as
  an implementation of synaptic projections\index{projections}.
  Connections will only handle a direct one-to-one transport from one
  application to another.  Re-mapping to actual receiving neurons,
  e.g. to implement an all-to-all projection, must be handled by one
  of the applications.  Thus, it may be better to regard the ports as
  \emph{proxy-objects}\index{proxy objects}, providing indirect access
  to neurons simulated by the other application.
\end{rationale}


\section{Timing Considerations}
\label{sec:timing}
  
Different applications may use different time steps and it is the
responsibility of MUSIC to ensure that data is delivered at the
appropriate time.  In order to minimize handshaking, both parts of a
connection pair locally calculate when the actual data transfer over
MPI takes place.  To ensure that these calculations produce
predictable results, simulation time is represented using integers
with a global micro-timestep\index{micro-timestep} common for all
applications.

Simulation time\index{simulation time} is local for each application
and MUSIC does not enforce unnecessary synchronization between these
local clocks.  Thus, an application producing data may be running
ahead of another application which consumes the same data.  MUSIC
internally builds a schedule which ensures that data arrives at the
appropriate local time in the receiving application.  Scheduling
becomes more complex when data is not only transferred in a
feed-forward manner, i.e. the connection graph contains loops.  In
this case MUSIC has to rely on the existence of sufficient delays in
the simulated model, typically corresponding to axonal
delays\index{axonal delay}.

\begin{figure}
  \begin{center}
    \begin{minipage}{0.45\textwidth}
      \includegraphics[width=\textwidth]{figures/ticklogic}
      \caption[Timing of data transfer, slowdown]{\label{fig:timingshorter}
        Transfer of data when sender has a shorter
        tick interval than the receiver}
    \end{minipage}
    \hfill
    \begin{minipage}{0.45\textwidth}
      \includegraphics[width=\textwidth]{figures/ticklogic2}
      \caption[Timing of data transfer, speedup]{\label{fig:timinglonger}
        Transfer of data when sender has a longer
        tick interval than the receiver}
    \end{minipage}
  \end{center}
\end{figure}

Figures~\ref{fig:timingshorter} and \ref{fig:timinglonger} illustrate
how MUSIC handles time when transferring continuous data over a connection.
In figure~\ref{fig:timingshorter}, the sender application uses a
shorter interval between the tick calls than the receiver.  The sender
side uses values sampled at the tick points to interpolate a value
corresponding to the point in time when the receiver makes its tick
call.

The dark middle area (labelled ``MPI'') is where the actual data
transfer takes place.  MUSIC makes use of the fact that the receiving
application can run with its simulation clock set independently of the
sender.  The arrows going ``backwards in time'' in this area reflect
the fact that the receivers clock is lagging.  This makes it possible
for data to arrive in time despite the fact that it was available
later than when it was arriving.

Figure~\ref{fig:timinglonger} illustrates what happens when the
receiver of continuous data is calling tick faster then the sender.
The sender will then buffer up values from the preceding and current
ticks and transfer this at a suitable tick call.  The receiver will
portion these values out by interpolating at the appropriate ticks.

The strategy of having the receiver application running with a delayed
local clock only works when the connection graph forms a directed
acyclic graph (DAG)\index{DAG}\index{acyclic graph}\index{loops}.
When loops occur it is necessary to allow for data arriving late, at
least somewhere along each loop.  MUSIC handles this via
\emph{acceptable latency}\index{acceptable latency}\index{latency}
which is a property of event input ports.  The receiving application
declares how late data may arrive, thus giving MUSIC room for
resolving the scheduling problem.  In the case of continuous data, the
application specifies a \emph{delay}\index{delay} which fulfills the
same purpose.


\section{Message Ports}\index{message port}

In addition to the port types which handle continuous and spike event
data, there are \emph{message ports}.  These allow for arbitrary
messages of, for example, control information being sent between
applications.  A multi-simulation may, for example, be controlled by a
script running in a Python process on one of the cluster nodes.  The
script may use a message port to alter a parameter or to turn on a
stimulus in an application at a certain point in time.

Messages sent from any process on the sender side are routed to all
processes on the receiver side which have announced there willingness
to receive messages.

To achieve independence between MUSIC applications, it is recommended
that the only messages being sent are text strings with the syntax of
the interpreter language of the receiving application, and that these
text strings originate from a user-specified configuration file read
by the sending application.


\section{Application Responsibilities}

One goal of MUSIC has been to limit the responsibilities imposed on
each application.  Here we present a step-by-step list of what an
application must do in order to participate in a multi-simulation.

\begin{enumerate}
\item \textbf{Initiate MUSIC}\index{initiate MUSIC}

  This is done by calling the \lstinline!setup! function.
\item \textbf{Create ports}

  Data to be imported and exported is identified by creating named
  ports.
\item \textbf{Map ports}\index{map ports}

  MUSIC is informed about where the actual data is located.
  This includes information about which processor owns each data
  element.  For continuous data it also includes information about
  where in memory it is stored, while for event data it defines what
  functions to use to send and receive the events.
\item \textbf{Initiate the runtime phase}

  At this stage, MUSIC can build the plan for communication between
  different processes.
\item \textbf{Advance simulation time}\index{tick}\index{advance time}
  \index{time}

  The application must call \lstinline!tick! at regular intervals
  to give MUSIC the opportunity to transfer data.
\item \textbf{Finalize MUSIC}\index{finalize}\index{terminate}

  By deleting the runtime object, all MUSIC communication is terminated.
\end{enumerate}


\chapter{Starting a Multi-Simulation}

\section{Overview}

Parallel programs based on MPI are normally started by running a
special program called \texttt{mpirun}\index{mpirun} (for MPI-1) or
\texttt{mpiexec}\index{mpiexec} (for MPI-2).  To start multiple
applications and enable them to communicate with each other, MUSIC
utilizes a special launcher program called \texttt{music}\index{music}
which, in turn, starts the different applications.  Information about
which applications should be started, and the communication pattern
between them is described in a common \emph{configuration
  file}\index{configuration file}.

\begin{rationale}
  The reason for not controlling configuration via the MUSIC API is
  that individual applications should remain ignorant about the
  structure of the full multi-simulation.  Thus, the API provides
  methods for asking about the parts of the configuration relevant for
  that application, i.e. the ports, but does not expose the complete
  communication graph.
\end{rationale}


\section{The Configuration File}

The main purpose of the configuration file is to control what
applications to start, and to connect output ports to input ports.
The configuration file specifies the number of processors allocated to
each application.

The configuration file consists of a sequence of blocks, each starting
with a non-indented bracket:

\begin{quote}
  [\emph{application\_label}]
\end{quote}

\noindent Each block consists of a sequence of configuration variable
definitions applying to one application.  The application label is
used to refer to ports of the application.  A variable definition
takes the form of an assignment:

\begin{quote}
  \emph{varname} = \emph{value}
\end{quote}

The following variable names have special meaning to MUSIC:
\begin{description}
  \item[binary] Pathname to application binary
  \item[args] Command line given to the binary
  \item[np] The number of MPI processes to allocate for the application
  \item[timebase] The length of a MUSIC micro-step, that is, the
    resolution of MUSIC:s internal clocks).  (Default value is 1 ns.)
\end{description}
Arbitrarily named parameters may also be included in the configuration
file and these parameters can be accessed from the applications.

A connection between a output and input port is specified using the
following syntax:
\begin{quote}
  \emph{application\_label.port\_name} \lstinline|->| \emph{application\_label.port\_name}
\end{quote}
\noindent.  The direction of the arrow (\lstinline|->|, \lstinline|<-|) indicates the
direction of data transport.  An output port can be connected to
multiple input ports while an input port can be connected to, at most,
one output port.

Optionally, the width of the connections between applications can be
specified:
\begin{quote}
  \emph{application\_label.port\_name} \lstinline|->|
  \emph{application\_label.port\_name} [\emph{width}]
\end{quote}
The application label can be omitted if it refers to the application
being specified by the surrounding block.
An example of a simple configuration file can be seen in
section~\ref{sec:conffile}.  Appendix~\ref{sec:specsyntax} specifies
the formal syntax of configuration files.

\begin{rationale}
  Information from the configuration file needs to be available both
  in order to launch the application binaries and during the setup
  phase.  Since launching must be done prior to MPI initialization, it
  is not possible to distribute configuration information via MPI
  itself.  In the reference implementation of MUSIC, environment
  variables are used to distribute this information to the
  applications.

  This information transfer is hidden within MUSIC, so a different
  implementation of MUSIC may use another technique.  In particular,
  if the applications are launched from a scripting program, such as
  PyNN\index{PyNN}, that program must also take care of transferring
  the relevant configuration information to the applications.
\end{rationale}


\chapter{Application Program Interface}

\section{Conventions}

This chapter describes the API to the MUSIC library.  The API is
object oriented and all communication with the library is performed
via instance methods of different classes of objects.  The most common
way of passing objects as arguments in MUSIC is via pointers.  The
only exception is the setup constructor.  The conventions are then
that it is the caller who should make sure that the object exists in
memory during the entire execution of the method, and it is the caller
who is responsible for the deallocation of the object afterwards.

\section{Error handling}

MUSIC follows the style of error handling in the MPI standard, which
is described in sections 7.2 and 7.3 in the MPI 1.1
report\cite{mpi1.1} and in section 2.8 of the MPI 2.0
report\cite{mpi2.0}.  This means that a MUSIC implementation may
choose not to handle some errors that occur during MUSIC calls.  Each
error handled by MUSIC generates a MUSIC exception.  MUSIC attempts to
fall back on the error handling mechanisms of MPI.  A MUSIC exception
thus results in a call to the MPI error handler.  MUSIC installs
suitable error codes, classes and strings so that the MPI error
handler is able to generate suitable error messages.  The default
error handler of MPI is \lstinline|MPI_ERRORS_ARE_FATAL| which means
that any error handled by MUSIC will result in the program being
aborted.

Using the error handling of MPI requires features only described in
the MPI 2.0 report.  For MPI implementations which lack this support,
MUSIC uses its own error handler which has the same behavior as
\lstinline|MPI_ERRORS_ARE_FATAL|.

\section{Setup}

\subsection{The setup constructor}

Each application initializes the MUSIC library through a call to the
setup constructor\index{setup}.  This constructor, in turn, calls
\lstinline|MPI::Init|\index{MPI::Init}\index{init} to initialize
MPI\index{initialize MPI}.  The setup constructor creates the setup
object through which the application can retrieve configuration
information, get an application wide communicator, and setup ports.

\begin{head}{setup}
  setup::setup (int& argc, char**& argv)
\end{head}
\begin{parameters}
  \lstinline|argc| &%
  reference to the \lstinline|argc| argument of \lstinline|main| \\
  \lstinline|argv| &%
  reference to the \lstinline|argv| argument of \lstinline|main| \\
\end{parameters}

This constructor must be called at most once; subsequent calls are
erroneous.  It accepts the \lstinline|argc| and \lstinline|argv| that are
provided by the arguments to \lstinline|main|.
\index{argc}\index{argv}

\begin{code}{Initializing MUSIC}
int main (int argc, char *argv[])
{
  MUSIC::setup* setup = new MUSIC::setup (argc, argv);

  /* parse arguments */
  /* rest of program */
}
\end{code}

\begin{rationale}
  The idea behind creating a specific setup object is to ensure that
  the application does not accidentally call functions relevant only
  for the setup phase at other times.
\end{rationale}


\subsection{Communicators}

During a multi-simulation the music library will create a unique
intra-communicator over the group of processes assigned to each
application.  This application wide communicator takes the role of the
global communicator \lstinline|MPI::COMM_WORLD| and is retrieved from
the setup object through a call to the \lstinline|communicator|
method.\index{communicator}

\begin{head}{communicator}
  MPI::Intracomm setup::communicator ()
\end{head}
\begin{parameters}
  \emph{return value} & the application wide communicator \\
\end{parameters}

The application is supposed to use the application wide communicator
in place of
\lstinline|MPI::COMM_WORLD|\index{MPI::COMM\_WORLD}\index{COMM\_WORLD}.

\index{rank}\index{Get\_rank}
\begin{code}{Accessing the application-wide communicator}
/* communicator with global scope */
extern MPI_Comm comm;

...
{
  ...
  comm = setup->communicator ();
  int rank = comm.Get_rank ();
  ...
}
\end{code}

\begin{rationale}
  An alternative to provide the \lstinline|communicator| function
  would have been to redefine \lstinline|MPI::COMM_WORLD|.  This would
  ensure that an application does not accidentally use the global
  communicator.  However, it may not always be possible to dynamically
  redefine this variable in all MPI implementations, so for the sake
  of portability, we have chosen a more straightforward technique.
\end{rationale}


\subsection{Port creation}

Ports\index{ports} are named sources (output ports) or sinks (input
ports) of data flows.  Output and input ports are distinct classes.
Ports are further subdivided into distinct classes depending on
whether they handle continuous or event data.

\index{publish\_output}\index{publish\_input}
\begin{head}{publish_cont_output,publish_cont_input,
            ,publish_event_output,publish_event_input,
            ,publish_message_output,publish_message_input}
  cont_output_port* setup::publish_cont_output (string id)

  cont_input_port* setup::publish_cont_input (string id)

  event_output_port* setup::publish_event_output (string id)

  event_input_port* setup::publish_event_input (string id)

  message_output_port* setup::publish_message_output (string id)

  message_input_port* setup::publish_message_input (string id)
\end{head}
\begin{parameters}
  \lstinline|id| & port name \\
  \emph{return value} & an unmapped port \\
\end{parameters}

Ports have two stages in life: the \emph{unmapped} stage and the
\emph{mapped} stage.  A port is unmapped when created.  The MUSIC
configuration file specifies connections between ports.  It is
possible to ask an unmapped port if it is connected, if it has a width
specified and, if so, what width it has.  A port becomes mapped when
its method \lstinline|map| is called.

\begin{code}{Creating an unmapped port}
cont_output_port* out =
   setup->publish_cont_output ("out");
\end{code}

\subsection{General port methods}

The port API includes methods to ask a port if it is connected, if it
has a width specified and what width it has if this width is
specified.

\subsubsection{Port connectivity}

The method \lstinline|is_connected| is used to check if the user has
specified a connection of this port to another port in the
configuration file.

\index{is\_connected}\index{connected port}
\begin{head}{is_connected}
  bool port::is_connected ()
\end{head}
\begin{parameters}
  \emph{return value} & \lstinline|true| only if connected\\
\end{parameters}

This method is typically used in cases where the use of some of the
ports of the application is optional.  In such a case, it is not
sensible to allocate any application resources to support the data
flow in question.  One example is if one wants to support output of
membrane potentials from a certain population of cells, but don't want
to waste resources if no one is listening.

\begin{code}{Optional handling of ports}
cont_output_port* out =
   setup->publish_cont_output ("Vm");
/* map port only if anyone is listening */
if (out->is_connected ())
  /* allocate application resources and map port */
\end{code}

\subsubsection{Port width}
\label{sec:width}

The width of a port\index{port width}, that is the number of data
elements transferred in parallel from a cont port or the largest
possible id of an event port $+ 1$, can be specified in the
configuration file.  This should be thought of as a request for a
given width.  Applications can use the method \lstinline|has_width| to
determine if a width has been specified and retrieve the width using
\lstinline|width|.  Message ports do not have width.

\index{has\_width}
\begin{head}{has_width}
  bool port::has_width ()
\end{head}
\begin{parameters}
  \emph{return value} & \lstinline|true| only if port width has been
                         specified \\
\end{parameters}

\index{width}
\begin{head}{width}
  int port::width ()
\end{head}
\begin{parameters}
  \emph{return value} & port width \\
\end{parameters}

\begin{rationale}
  Applications can use the above methods to adapt their port width.  A
  typical usage would be a general purpose post-processing tool which
  receives information from an ongoing simulation.  Such a tool can
  publish a number of optional input ports and then use
  \lstinline|is_connected| and \lstinline|width| to adapt its internal
  processing depending on what kind of data source it is actually
  connected to.  See example \ref{code:adaptivewidth}.
\end{rationale}


Some applications, for example those who provide some kind of more
generic processing, like a visualization application, may want to be
able to accept input of any width.  This can be achieved using the
methods described in section \ref{sec:width}:

\begin{code}{Publishing port of adaptive width\label{code:adaptivewidth}}
{
  ...
  /* Publishing a port of adaptive width */
  double* state_vars;
  MUSIC::cont_input_port* in =
     setup->publish_cont_input ("in");
  if (!in->has_width ())
    /* report error */
  else
    {
      int size = in->width ();
      /* for clarity we assume that n_elements
         is a multiple of size */
      int n_local = n_elements / size;
      /* example continues as in next example */
      ...
      
    }
}
\end{code}


\subsection{Mapping cont ports}
\index{cont ports}

A port is informed about what data exists locally and how to access it
by mapping it.  Cont ports transfer data from or to memory during
\lstinline|tick| calls and need to know the layout of data in
memory.  This information is captured by the data map argument
\lstinline|dmap|.  The \lstinline|cont_data| type is described in
section \ref{sec:datamap} below.


\index{map}\index{mapping cont ports}
\begin{head}{map}
  void cont_output_port::map (cont_data* dmap,
                              int max_buffered)

  void cont_input_port::map (cont_data* dmap,
                             double delay,
                             int max_buffered,
                             bool interpolate)
\end{head}
\begin{parameters}
  \lstinline|dmap| & the data map associated with the port \\
  \lstinline|delay| & delay of data arrival in simulation time (s) \\
  \lstinline|max_buffered| & maximal amount of data buffered (ticks)
  \\
  \lstinline|interpolate| & enable interpolation (boolean) \\
\end{parameters}

The optional argument \lstinline|delay| informs MUSIC of when,
according to simulation time, to sample data on the sender side.  If
enabled, linear interpolation is used to obtain an approximation of
the state at this time.  The default delay is zero.  Delayed continuous
data may be used in connectionist networks when modeling brain
pathways.  A delay is \emph{required} at some point when communicating
continuous data in a loop.

Buffering data in output and input ports gives more efficient
communication since data can be sent fewer times in larger packets.
By default MUSIC buffers some reasonable amount of data.  In certain
situations it is necessary to be careful about memory usage.  Using
the optional argument \lstinline|max_buffered| the application can
give MUSIC a bound on how much data to buffer.  MUSIC decides how much
data to buffer based on the lowest \lstinline|max_buffered| parameter
given when mapping each of a set of connected ports and on latency
considerations when applications are connected in loops.  A
\lstinline|max_buffered| value of \(N\) ticks means: don't buffer more
data than is sufficient for communicating at every \(N\)\,th tick.

When the optional argument \lstinline|interpolate| is
\lstinline|true|, MUSIC uses linear interpolation to determine the
values delivered on the receiver side.  This is the default behavior.
By passing \lstinline|false| this interpolation can be switched off in
which case MUSIC selects the sample on the sender side which is
closest according to simulation time.



\begin{code}{Mapping ports to internal data\label{code:mapping}}
{
  ...
  int size = comm.Get_size ();
  int rank = comm.Get_rank ();
  /* for clarity we assume that n_elements
     is a multiple of size */
  int n_local = n_elements / size;
  double* state_vars = new double[n_local];
  MUSIC::cont_input_port* out =
     setup->publish_cont_output ("out");
  MUSIC::array_data dmap (state_vars, MPI::DOUBLE,
                          rank * n_local, n_local);
  out->map (&dmap);
  ...
}
\end{code}


\subsection{Mapping event ports}
\index{event ports}

\index{map}\index{mapping event ports}
\begin{head}{map}
  void event_output_port::map (index_map* indices,
                               int max_buffered)

  void event_input_port::map (index_map* indices,
                              event_handler* handle_event,
                              double acc_latency,
                              int max_buffered)
\end{head}
\begin{parameters}
  \lstinline|indices| & the index map associated with the port \\
  \lstinline|handle_event| & a user-defined event handler \\
  \lstinline|acc_latency| & acceptable latency for incoming data (s) \\
  \lstinline|max_buffered| & maximal amount of data buffered (ticks) \\
\end{parameters}

Since event ports don't access data the same way as cont ports, they
do not require a full \lstinline|data_map|.  Events are communicated
to the application through an \emph{event handler}\index{event
handler}.  The event handler is called by MUSIC when the application
calls \lstinline|tick|.  It is called once for every spike delivered.

Some spiking neural network models include axonal delays.  The MUSIC
framework assumes that handling and delivery of delayed spikes occurs
on the receiver side.  In such a case, the receiver may allow MUSIC to
deliver a spike event later than its time stamp according to local
time.  The maximal acceptable latency is specified through the
\lstinline|acc_latency|\index{acc\_latency} argument.

The optional argument \lstinline|max_buffered|\index{max\_buffered}
has a similar meaning as for event ports above but the actual amount
of data buffered is, in this case, not deterministic since it is
dependent on spike rate.

\subsubsection{Sending events}
\index{sending events}

The sender registers an event for transmission by calling the method
\lstinline|insert_event|.

\index{insert\_event}
\begin{head}{insert_event}
  void event_output_port::insert_event (double t, int id)
\end{head}
\begin{parameters}
  \lstinline|t| & trigger time of the event (s) \\
  \lstinline|id| & the sender local index \\
\end{parameters}

MUSIC guarantees that this event will be delivered through a call to
the user-specified \lstinline|event_handler| on the receiver side no
later that the acceptable latency relative to receiver local time.
The time \lstinline|t| must be between the simulation time of the last
tick and that of the next.


\subsubsection{Receiving events}
\index{receiving events}

\index{event\_handler}
\begin{head}{event_handler,operator}
  class event_handler {
  public:
    virtual void operator () (double t, int id) = 0;
  };
\end{head}
\begin{parameters}
  \lstinline|t| & trigger time of the event (s) \\
  \lstinline|id| & the receiver local index \\
\end{parameters}

Event handlers are called by event input ports to deliver events.  The
application is supposed to customize \lstinline|event_handler| by
subclassing it.


\subsection{Mapping message ports}
\index{message ports}

Message handlers are called by message input ports to deliver
messages.  The application is supposed to customize
\lstinline|message| and \lstinline|message_handler| by subclassing.
It is recommended that messages are text strings with the syntax of
the interpreter language of the receiving application, and that these
text strings originate from a user-specified configuration file read
by the sending application.

\index{map}\index{mapping message ports}
\begin{head}{map}
  void message_output_port::map (int max_buffered)

  void message_input_port::map (message_handler* handle_message,
                                double acc_latency,
                                int max_buffered)
\end{head}
\begin{parameters}
  \lstinline|handle_message| & a user-defined message handler \\
  \lstinline|accept_latency| & acceptable latency for incoming data (s) \\
  \lstinline|max_buffered| & maximal amount of data buffered (ticks) \\
\end{parameters}

Message ports behave similarly to event ports in that events are sent
and delivered using similar mechanisms, but while events are routed
between processes based on event id, messages are routed to all
processes on the receiver side which have provided a
\lstinline|message_handler| to \lstinline|map|.  All arguments to
\lstinline|map| for message ports are optional.

\subsubsection{Sending messages}

The sender registers a message for transmission by calling the method
\lstinline|insert_message|.

\index{insert\_message}
\begin{head}{insert_message}
  void message_output_port::insert_message (message* msg, size_t size)
\end{head}
\begin{parameters}
  \lstinline|msg| & pointer to message subclass instance \\
  \lstinline|size| & size of message instance in bytes \\
\end{parameters}

MUSIC will deliver this message through a call to the user-specified
\lstinline|message_handler| on the receiver side.


\subsubsection{Receiving messages}

\begin{head}{message_handler,operator}
  class message {
   public:
    message (double t);
    double t ();
  };

  class message_handler {
  public:
    virtual void operator () (message* msg, size_t size) = 0;
  };
\end{head}
\begin{parameters}
  \lstinline|msg| & pointer to message subclass instance \\
  \lstinline|size| & size of message instance in bytes \\
\end{parameters}


\subsection{Index maps}
\index{index maps}

An index map is a mapping from local data element indices to
global. An index map instance thus holds information of which global
indices belong to the local MPI process and of their order with regard
to local index.  The most general form is the
\lstinline|permutation_index| which allows for an arbitrary mapping.

\index{permutation\_index}
\begin{head}{permutation_index}
  permutation_index::permutation_index (int* indices,
                                        int size)
\end{head}
\begin{parameters}
  \lstinline|indices| & vector of global indices \\
  \lstinline|size| & number of global indices \\
\end{parameters}

\index{linear\_index}
\begin{head}{linear_index}
  linear_index::linear_index (int baseindex, int size)
\end{head}
\begin{parameters}
  \lstinline|baseindex| & global index of first local element \\
  \lstinline|size| & number of contiguous global indices \\
\end{parameters}

When a cont output port is mapped it becomes associated with a set of
state variables (or other data) in the memory of the sender.  When the
receiver calls \lstinline|runtime::tick|, an estimate of the values
of these variables are stored in a set of variables associated with an
input port on the receiver side.  Similarly, an event output port is
mapped to a set of event id:s.

While the number of variables or id:s on the receiver side is always
the same as on the sender side, the data can be distributed in
different ways between MPI processes on the sender side compared to
the receiver side.  In fact, sender and receiver may consist of
different numbers of processes.

Index maps are used in each MPI process to tell MUSIC how data is
distributed and ordered by enumerating the global indices represented
by the process in local order.

\subsection{Data maps}
\label{sec:datamap}
\index{data maps}

A data map encapsulates how a port accesses its data.  While an index
map is a mapping between two kinds of indices, the data map also
contains information about where in memory data resides, how it is
structured, and, the type of the data elements.  One type of data map
is the array data map, which describes arrays of data elements.
See example \ref{code:mapping}.

\index{array\_data}
\begin{head}{array_data}
  array_data::array_data (void* buffer, MPI_Datatype type,
                          index_map* map)
\end{head}
\begin{parameters}
  \lstinline|buffer| & data memory location \\
  \lstinline|type|   & data type \\
  \lstinline|map|    & index map \\
\end{parameters}

Since data organized in arrays is common, MUSIC provides a convenience
form of the array data map constructor which also creates an index
map:

\begin{head}{array_data}
  array_data::array_data (void* buffer,
                          MPI_Datatype type,
                          int  baseindex,
                          int size)
\end{head}
\begin{parameters}
  \lstinline|buffer|    & data memory location \\
  \lstinline|baseindex| & global index of first local element \\
  \lstinline|size|      & number of contiguous global indices \\
\end{parameters}


\subsection{Configuration variables}
\index{configuration variables}

The values of all variables defined in the configuration file can be
queried using the method \lstinline|config|.

\index{config}
\begin{head}{config}
  bool config (string name, string* result)

  bool config (string name, int* result)

  bool config (string name, double* result)
\end{head}
\begin{parameters}
  \lstinline|name|     & variable name \\
  \lstinline|result|  & pointer to location where result should go \\
  \emph{return value} & true if value of correct type was found \\
\end{parameters}

Querying for a value of type \lstinline|int| or \lstinline|double|
expects a value of the correct type, if defined in the configuration
file.  If the variable is defined, but its value can't be translated
into the correct type this causes an error condition.

\begin{code}{Querying configuration variables}
/* Retrieving the parameter gKCa
   from configuration file */
double gKCa;
if (!config ("gKCa", &gKCa))
  gKCa = 29.5e-9; // default value
\end{code}

\section{Runtime}

\subsection{The runtime constructor}

\begin{head}{runtime}
  runtime::runtime (setup* s, double h)
\end{head}
\begin{parameters}
  \lstinline|s| & pointer to the setup object \\
  \lstinline|h| & simulated time elapsed between each tick (s) \\
\end{parameters}

Creation of the runtime object marks the transition from the setup to
the runtime phase.  The runtime object constructor destroys the setup
object, effectively making it impossible to create new ports.

\begin{code}{runtime}
  ...
  MUSIC::runtime runtime = MUSIC::runtime (setup, stepsize);
  ...
\end{code}

\begin{rationale}
  The step size is given as a real number (in seconds) since this
  makes most sense from the applications point of view.  Internally,
  this number is converted to an integer (using the time micro step
  time base).  This is made to ensure that all processes use exactly
  the same numbers even when the multi-simulation is running on mixed
  architectures.  Both sides of a connection must agree on at what
  ticks data is transferred over the MPI connector to minimize the
  need for handshaking during the runtime phase.
\end{rationale}

\begin{rationale}
  In order to create a deterministic schedule for buffering and data
  transfer, we require that \lstinline|tick| is called at fixed
  intervals in simulated time.  We realize that some applications may
  use a variable time step\index{variable time step} for their
  numerical integrations, which may then make it harder to execute
  these tick calls at the right time.  However, allowing variable tick
  steps would have made it impossible to use a pre-computed
  deterministic schedule and enforced repeated handshaking throughout
  the runtime phase, resulting in a substantial performance
  degradation.

  Note that the tick step does not need to be equal to the internally
  used integration step\index{integration step}.  We believe that most
  large scale parallel simulators already have some means for fixed
  interval operations, e.g. to handle logging to files or graphics,
  which may be utilized also for the tick calls.
\end{rationale}


\subsection{The tick}
\index{tick}

\begin{head}{tick}
  void runtime::tick ()
\end{head}
\begin{parameters}
\end{parameters}

The tick function must be called at regular intervals in simulation
time.  The application chooses the interval as a parameter to the
\lstinline|runtime| constructor, normally based on the time step used
in the application.  The \lstinline|tick| function is typically called
in the main simulation loop of each application.  Different
applications may use different tick-intervals and MUSIC will ensure
that time is incremented consistently throughout the multi-simulation.

The interface may, or may not, exchange data with other applications
at the tick call.  The application must ensure that exported data
values are valid when \lstinline|tick| is called.  It must also expect
that imported values may change and that event and message handlers
are called.

\begin{rationale}
  The idea behind the \lstinline|tick| call is to hide the complexity
  of data buffering and MPI transfer from the application.  For
  efficient data transfer, MUSIC will try to buffer data both at the
  sending and receiving port in order to send data in large chunks.
  Internally, MUSIC will use a pre-computed schedule to keep track of
  at what ticks the actual data transfer should take place and when
  data should instead be buffered for later transfer.
\end{rationale}


\subsection{Simulation time}
\index{simulation time}

The method \lstinline|time| returns local time in seconds.

\index{time}
\begin{head}{time}
  double runtime::time ()
\end{head}
\begin{parameters}
  \emph{return value} & local time (s) \\
\end{parameters}

\lstinline|time| returns the local time of last \lstinline|tick|
call.  Time starts at 0\,s.  While it is possible, and recommended, to
let MUSIC keep track of time for the application, this is not
required.

\begin{rationale}
  To schedule data transfers, MUSIC needs to keep track of the
  simulation time of all applications via its internal integer
  representation.  If the application independently manages its own
  clock, typically by incrementing a floating point variable, there is
  a risk for drift between the two time representations.  The
  \lstinline|time| function makes it possible for the application to
  keep its clock in perfect synchronization with time in the other
  applications.
\end{rationale}


\subsection{Finalization}
\index{finalize}

An application supporting MUSIC should replace its call to
MPI::Finalize with the destruction of the MUSIC runtime object.

\begin{code}{}
  ...
  MUSIC::runtime runtime = MUSIC::runtime (setup, stepsize);
  ...
  delete runtime;
\end{code}

\chapter{A Complete Example}

This chapter shows a minimal but still complete example.  It consists
of two applications, \texttt{waveproducer} and \texttt{waveconsumer},
and a configuration file used to launch and connect them.


\section{Configuration File}
\label{sec:conffile}

The configuration file starts the waveproducer application on three
processors and waveconsumer on four.

\lstinputlisting[language=Clean,frame=single]{../test/wavetest.music}


\section{Data Generating Application}

\lstinputlisting{../test/waveproducer.cc}


\section{Data Consuming Application}

\lstinputlisting{../test/waveconsumer.cc}


\chapter{Adapting Existing Applications}

In this chapter we will highlight the steps necessary to adapt an
existing neural simulator to MUSIC.  We will assume that the simulator
is already using MPI to simulate large networks of interconnected
neurons.

The two main tasks that need to be handled are: firstly, to create and
map ports for data to be imported and exported, and, secondly, to
ensure that the \lstinline|tick| function is called at regular
intervals.


\section{Creating and Mapping of Ports}

The application needs to inform MUSIC about what data to import and
export, and where this data resides.  A simulator will typically use
some sort of script files where the user specifies the model and other
aspects of the simulation.  If possible, it is desirable to extend
these scripts such that the user can also specify what model variables
to communicate, and what names to use as identifiers.

Assuming that we have introduced such constructs into the scripting
language, we must decide on a suitable point in the initialization
process where ports should be created and mapped.  Since continuous
data is read from, or written to, arrays in memory, the program must
have allocated its runtime data structures in order to perform the
mapping.

Communication of spikes will use event ports, where functions are used
to send and receive spikes.  Sending of spikes is relatively
straightforward, since the only thing needed is to add a call where
spikes are normally detected in the program.  Receiving spikes
requires more administration, since the spikes can be received earlier
than when they should reach their destination compartment.  It is
therefore necessary to save incoming spikes in some sort of sorted
buffer (typically a priority queue).

\begin{figure}
  \begin{center}
    \begin{minipage}[t]{0.45\textwidth}
      \includegraphics[width=\textwidth]{figures/remapping}
      \caption[Processing of incoming data]{\label{fig:remapping1}
        The sender application presents the data to the output port in
        the same order as it is stored internally.  The receiving
        application will see the transferred data in the same order
        and will explicitly have to implement a proper reordering to
        implement a typical synaptic projection.
      }
    \end{minipage}
    \hfill
    \begin{minipage}[t]{0.45\textwidth}
      \includegraphics[width=\textwidth]{figures/remapping2}
      \caption[Remapping of data within MUSIC]{\label{fig:remapping2}
        If there is a one-to-one correspondence between sending and
        receiving neurons, the receiving application can specify an
        appropriate index map to instruct MUSIC to send the data
        directly to the right destination.
      }
    \end{minipage}
  \end{center}
\end{figure}

In addition, MUSIC will always present the spikes as they appear in
the sending group of neurons.  In most situations, the receiving
application will want to implement a remapping to the target
compartments, as illustrated in figureref{fig:remapping1}.  One spike
may thus end up at multiple postsynaptic compartments, spread out over
the processors of the receiving application.

In some situations it may be desirable for the receiving application
to avoid the remapping.  The application can the utilize the different
forms of mappings available in MUSIC to create a general permutation
so that MUSIC will send the spikes directly to the processor where it
should be handled.  This situation is illustrated in
figure~\ref{fig:remapping2}


\section{Advancing Simulation Time}

The application must call the \lstinline|tick| function repeatedly
throughout the simulation.  The application will have to ensure that
these calls are made at regular intervals, as specified to the runtime
constructor.  Note that this refers to \emph{simulated time}; there is
no need to consider how long computation time (``wall clock time'') is
used between tick calls.

If the application makes use of variable time steps internally, it may
be necessary to use some sort of checkpoints at fixed intervals where
tick can be called.  It is not necessary to call tick at every
integration step, but the calls should not be too infrequent.

The tick calls are the only times during runtime when MUSIC will use
MPI.  MUSIC will then use its own communicators, not to interfere with
the MPI operations of the application.  Still, we recommend that the
application does not intersperse the tick calls with ongoing MPI
operations.


\section{Initialization and Finalization}

\subsection{Initiate MUSIC}\index{initiate MUSIC}

The idea here is to replace the call to \lstinline|MPI:Init| with a
call to the \lstinline|MUSIC:setup| constructor.  The setup
constructor calls \lstinline|MPI:Init| for the application.

The setup object is used to query about configuration information, to
get the local communicator, and to create ports.

The application will have to replace all uses of the global
communicator \lstinline|MPI::COMM_WORLD| with the communicator
supplied by MUSIC.  The global communicator will be global over all
applications and it is necessary to limit the MPI operations to be
local within the application.

There should be no need to link an application differently when it is
used together with other applications in a MUSIC setting compared to
when it is used in a stand-alone setting.  In order to support
``standard'' operation for the application,
\lstinline|setup::communicator ()|, therefore, will return
\lstinline|MPI::COMM_WORLD| if the job is started directly with
\lstinline|mpirun| instead of with the MUSIC launcher.



\subsection{Initiate the runtime phase}

Creating the runtime object will implicitly call the \lstinline|setup|
object destructor to ensure that the application will no longer be
able to change the communication pattern.  At this stage, MUSIC can
build the plan for communication between different processes.

\subsection{Finalize MUSIC}\index{finalize}\index{terminate}

The application should also replace its call to
\lstinline|MPI:Finalize| by the destruction of the runtime object.
The destructor method will internally call \lstinline|MPI:Finalize|.

\bibliographystyle{unsrtnat}
\bibliography{music-rfc}

\appendix

\chapter{C Interface}

In the final MUSIC specification, this appendix will contain a
description of the C API to MUSIC.


\chapter{Specification File Syntax}
\label{sec:specsyntax}

\newcommand{\nt}[1]{$<$#1$>$}

\begin{tabular}{lcl}
\nt{simulation spec}   & ::= & \{ \nt{application block} \} \\
\nt{application block} & ::= & \nt{newline} '[' \nt{application id} ']' \{ \nt{declaration}
\} \\
\nt{application id}    & ::= & \nt{symbol} \\
\nt{declaration}       & ::= & \nt{variable def} $|$ \nt{connection} \\
\nt{variable def}      & ::= & \nt{variable} '=' \nt{value} \\
\nt{variable}	       & ::= & \nt{symbol} \\
\nt{value} 	       & ::= & \nt{integer} $|$ \nt{float} $|$ \nt{string} \\
\nt{connection}	       & ::= & \nt{port id} \nt{direction} \nt{port id} [ \nt{width} ] \\
\nt{port id}	       & ::= & \nt{application id} '.' \nt{port} $|$
\nt{port} \\
\nt{port}	       & ::= & \nt{symbol} \\
\nt{direction}	       & ::= & $->$ $|$ $<-$ \\
\nt{width}	       & ::= & '[' \nt{integer} ']' \\
\end{tabular}

\printindex

\end{document}


%%% Local Variables: 
%%% mode: latex
%%% TeX-master: t
%%% eval: (flyspell-mode 1)
%%% eval: (ispell-change-dictionary "american")
%%% eval: (flyspell-buffer)
%%% End: 
