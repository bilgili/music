\documentclass[a4paper]{report}

\usepackage[utf8]{inputenc}
\usepackage{fancyhdr}
\usepackage{color}
\usepackage{graphicx}

%%%%% Formatting %%%%%

% Use the metatext environment around text that should not appear in
% the final document
\newenvironment{metatext}%
{\color{blue}}%
{}


% Use the rationale environment around arguments for design decisions
\newenvironment{rationale}%
{\par\paragraph{Rationale:}}%
{\par}


\pagestyle{fancy}


%%%%% Actual content starts here %%%%%
\begin{document}

\title{MUSIC --- Multi-Simulation Coordinator\\
  Provisional Specification}

\author{Örjan Ekeberg and Mikael Djurfeldt}

\maketitle

\begin{abstract}
  MUSIC is an API allowing large scale neuron simulators using MPI
  internally to exchange data during runtime.
\end{abstract}


\tableofcontents

\chapter{Introduction}

\section{Scope}

Music is a standard for large scale simulators to exchange data at
run-time.  The standard is designed specifically for interconnecting
neuronal simulators, either with each-other or with other tools.


\subsection{Portability}

The interface and support software should be written to support most
high-end hardware that is used and will be used for large scale
neuronal modelling in the foreseeable future.


\subsubsection{Based on MPI}

Most, if not all, current efforts in large scale neuronal simulations
are based in the MPI standard.  Music will use MPI to run the
different simulators, and must allow each simulator to use MPI
internally without interfering with each other.

MPI encapsulates software optimizations for specific hardware.  By
basing the interface on MPI, we can benefit from such optimizations.


\subsubsection{Possible to run on BG/L}

BG/L nodes do not support multiple parallel tasks.  It's e.g. not
possible to start both Neuron and Genesis on the same node.


\subsection{Easy to use and to adapt existing software}

It must be possible to adapt existing simulators so that they can
participate in a multi-simulation without too much effort.  We rely in
the simulator developers to make these adaptations.

It should also be simple to write new applications to facilitate the
development of general purpose tools.  For example, a researcher can
develop a tool for calculating synthetic EEG from simulation data.
Via Music, this tool should then be useful for anybody using any
neuronal simulator which supports the common interface.


\section{Concepts}

We need to support efficient transfer of spike events and continuous
values.  How should these be handled in the API?  What can be expected
and utilized from the native data representation in existing programs?

Each module will communicate data via \emph{ports}.  These are either
\emph{imports} or \emph{exports}.


\chapter{Running a music simulation}

\chapter{Application Program Interface}

\section{Launching}

We will have to provide tools for starting the simulators.  This
procedure will also take care of initializing communication between
processes.  Since most simulators will use parallel processing also
internally, it is necessary for the simulator to rely in the interface
for some of its own initialization too.  The tools should therefore be
useful for running simulators even in situations when only one
simulator is running.


\subsection{execve}

\begin{metatext}
  MPI-2 comes with a program mpiexec which is capable of starting
  multiple applications directly.  In MPI-1 one can achieve the same
  result by starting a small "launcher" program which uses execve to
  start the real applications.  We should be able to write our interface
  so that it works in both situations.
\end{metatext}

\subsection{MPI::Init}

Each application will have to call an initialization routine in the API
which in turn initializes MPI.


\subsection{Communicators}

One communicator for each application, given to the application from
the API.

We should probably use the global communicator for all
inter-application communication.

\begin{metatext}
  \paragraph{Consideration:} Check if there are any advantages of having one
  separate communicator for each communication channel.
\end{metatext}


\section{Setup}

\paragraph{Idea:} initialization call returns a \emph{setup object}
which has methods for binding and other types of setup specification
operations.  The setup phase ends via a setup-done call which consumes
the setup object (to prevent further setup operations), and returns a
run-time object.  The run-time object contains the API for
synchronization (tick) and data transfer operations.

Both sender and receiver should be able to choose what variables to
transfer.


\section{Runtime}

\subsection{tick ()}

Called in the local simulation loop of each application.
The interface may, or may not, exchange data with other applications
at the tick call.

The application must ensure that exported data values are valid when
tick() is called.  It must also expect that imported values may change.

The tick() is called at regular intervals in simulation time.
The application chooses the interval, normally based on the time step
used in the application.  The interface must handle that different
applications can use different tick-intervals.


\subsection{Aliasing}

\begin{figure}
  \begin{center}
    \begin{minipage}{0.45\textwidth}
      \includegraphics[width=\textwidth]{figures/ticklogic}
      \caption{Transfer of data when sender has a shorter
        tick interval then the receiver}
    \end{minipage}
    \hfill
    \begin{minipage}{0.45\textwidth}
      \includegraphics[width=\textwidth]{figures/ticklogic2}
      \caption{Transfer of data when sender has a longer
        tick interval then the receiver}
    \end{minipage}
  \end{center}
\end{figure}

The interface should take care of spatial and temporal aliasing, that
is, to interpolate in space and time when the space and time steps of
two applications do not match.

To minimize communication, we always want to do the conversion on the
side which has the most dense data.

\paragraph{Motivation:} The application or model should not have to be
adapted to the steps used in other applications.


\subsubsection{Filters on sender side}

When the receiver of a data-flow has a longer tick-interval than the
receiver, then filters can be engaged to compress, smoothen, average,
or otherwise aggregate the data before transmission.

\begin{rationale}
  Filtering is needed on the sender side.  If we connect an existing
  simulation with a receiver with longer time step, we don't want to
  rewrite the simulation code to perform filtering.  Thus, the
  interface must support filtering.
\end{rationale}


\subsubsection{Interpolation on receiver side}

Needed when sender is slower than receiver.

Note that to make it possible to do interpolation, the sender has to
produce the next value before the receiver can enter the intermediate
time interval.  This needs careful scheduling of the transmissions to
avoid dead-lock situations.

\section{Specification file syntax}

\newcommand{\nt}[1]{$<$#1$>$}

\begin{tabular}{lcl}
\nt{simulation spec}   & ::= & \{ \nt{application block} \} \\
\nt{application block} & ::= & \nt{newline} '[' \nt{application id} ']' \{ \nt{declaration}
\} \\
\nt{application id}    & ::= & \nt{symbol} \\
\nt{declaration}       & ::= & \nt{variable def} $|$ \nt{connection} \\
\nt{variable def}      & ::= & \nt{variable} '=' \nt{value} \\
\nt{variable}	       & ::= & \nt{symbol} \\
\nt{value} 	       & ::= & \nt{integer} $|$ \nt{float} $|$ \nt{string} \\
\nt{connection}	       & ::= & \nt{port id} \nt{direction} \nt{port id} [ \nt{size} ] \\
\nt{port id}	       & ::= & \nt{application id} '.' \nt{port} $|$
\nt{port} \\
\nt{port}	       & ::= & \nt{symbol} \\
\nt{direction}	       & ::= & $->$ $|$ $<-$ \\
\nt{size}	       & ::= & '[' \nt{integer} ']' \\
\end{tabular}

\end{document}


%%% Local Variables: 
%%% mode: latex
%%% TeX-master: t
%%% End: 
